\documentclass[a4paper]{article}

\title{BetaBoard Datasheet \texttt{Preliminary}}
\author{Harry Beadle, George Brown}

\usepackage{graphicx}
\usepackage{float}
\usepackage[margin=1in]{geometry}

\begin{document}

\maketitle

\section*{18 Strip APA102 Multiplexer and Driver over RS485}

\section{Features}
\begin{itemize}
\item Drives up to 18 Strips of 255 LEDs, totaling up to 4590 APA102 devices.
\item Takes input using the RS485 protocol.
\item Supports up to 40A continuous current driving the LEDS, with the ability to support much higher output currents with auxiliary power supplies.
\end{itemize}

\begin{figure}[H]
\centering
\includegraphics[width=\textwidth]{assets/block_diagram.png}
\caption{Block Diagram}
\end{figure}

\subsection{Protocols}\label{protocols}

\subsubsection{RS485 Data Packet Format}\label{rs485-data-packet-format}

The RS485 Packet is sent from the compute platform to the BetaBoard. It
contains the address of the board, the strip for which the LED data is
for and the LED data.

\begin{center}
\begin{verbatim}
ID[8] STRIP[8] DATA_LENGTH[8] LED1[24] LED2[24] ... LEDN[24]
\end{verbatim}

\begin{table}[H]
\begin{tabular}{lll}
Tag & Bits & Description \\
\hline
\texttt{ID} & 8 & Board ID  \\
\texttt{STRIP} & 8 & Strip Address  \\
\texttt{DATA-LENGTH} & 8 & Number of LEDs to be set from packet \\
\texttt{LED}$n$ & 24 & Three bytes for each color of each LED, in the format \texttt{BGR}
\end{tabular}
\end{table}
\end{center}

As an example setting 1 LED in strip \texttt{0x00} on board ID \texttt{0x00} to \texttt{0xFFFFFF}:

\begin{verbatim}
ID------- STRIP---- DATA_LEN- LED1-------------------------
0000 0000 0000 0000 0000 0001 1111 1111 1111 1111 1111 1111
\end{verbatim}

\end{document}